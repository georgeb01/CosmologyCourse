\documentclass[11pt]{article}
    \renewcommand{\baselinestretch}{1.05}
    \usepackage{amsmath,amsthm,verbatim,amssymb,amsfonts,amscd, graphicx}
    \usepackage{graphics}
    
    \topmargin0.0cm
    \headheight0.0cm
    \headsep0.0cm
    \oddsidemargin0.0cm
    \textheight23.0cm
    \textwidth16.5cm
    \footskip1.0cm
    \newcommand{\inchsign}{$^{\prime\prime}$}
    
     \begin{document}
     
    \title{PHY306/406}
    \author{Dr. James Mullaney}
    \maketitle
    
    \section{Essential Information}
    \begin{itemize}
    
    \item Lectures:
    \begin{itemize}
        \item 1pm Monday in Diamond Workroom 1
        \item 11am Thursday in Diamond Lecture Theatre 9.
    \end{itemize}

    \item Current course material available at:

    \noindent
    \url{https://github.com/jrmullaney/CosmologyCourse}

    \item Previous years' course material available at:

    \noindent
    \url{http://www.hep.shef.ac.uk/cartwright/phy306/index.html}

    \item Textbook:
    
    \noindent
    ``Introduction to Cosmology'' by Barbara Ryden, published by Cambridge University Press
\end{itemize}

    \section{The Course}
    The aim of this course is to provide you with an understanding of the Universe as its own entity. That is to say that the focus of the course is less on the {\it contents} of the Universe (e.g., stars, galaxies, plantes, people), but the Universe itself. In the first part of the course, we will explore how the Universe has expanded over cosmic time, and what dictates the rate of this expansion. Next, we will cover how astromers and cosmologists measure this expansion and place constraints on the parameters that describe our models of the Universe. Later, we will try to gain insights into the Dark Universe (dark matter and energy), before delving into the very earliest fractions of seconds of the Universe. Toward the end of the course, we will take a look at how the Universe has evolved from those first few moments to produce the huge variety of astronomical objects that we see today. 
    
    The course will consist of 21 lecture sessions (but some of these will be revision lectures) and 3 problems classes. There will be 14 lectures and one problems class before the Easter break. 

    \section{Assessment}
    Assessment depends on whether you are taking PHY306 or PHY406:
    \begin{itemize}
    \item PHY306: Assessment is in the form of two class tests (10\% each) and a final exam (80\%).
    \item PHY406: Assessment is in the form of two class tests (7.5\% each), a reading exercise (10\%), and a final exam (75\%). The reading exercise consists of reading a review article and answering a set of questions on the article. It is the answers to the questions that are marked.
    \end{itemize}

\noindent
 There will be one class test in the final week before Easter on Monday, 19th March. The other class test will be held after the Easter break. The second class test will be held after the Easter break at 2pm on Monday, 30th April.

\noindent
The deadline for the PHY406 reading exercise is 5pm on Friday, 4th May.

\noindent
    The rubric for the exam is: 
    
    \begin{itemize}
        \item Answer ALL questions in Section A (compulsory) and TWO questions in Section B.
        \item Section A consists of four short questions marked out of 5 (a total of 40\% of the exam)
        \item Section B will include four questions marked out of 15 (choose any two; each is worth 30\% of the exam).
    \end{itemize}

    \section{Learning Material}
    There are five different sources of learning material for this course. Three of these I have prepared myself, one is provided by the person who previously delivered this course - Dr. Susan Cartwright - and the final one is the textbook ``Introduction to Cosmology'' by Barbara Ryden, published by Cambridge University Press. The material I have prepared is almost wholly derived from that textbook.
    
    The three types of learning material I have produced for the course are:
    \begin{itemize}
    \item A set of typed notes to accompany each lecture. These are what I refer to when writing on the board and are thus written in note form (e.g., largely single sentence bullet points, equations etc.). There are no ``additional'' notes that I secretly keep to myself.
    \item Lecture slides. You will have access to all the lecture slides that I present throughout the course.
    \item Python Notebooks. Whenever I produce a plot (i.e., graph) or any data that appears in any of the lectures, I'll make the Python code I used to produce that plot/data available to you.

    \end{itemize}

    \noindent
    All of the above material will be made available via the course's github pages:

    \noindent
    \url{https://github.com/jrmullaney/CosmologyCourse}
    
    Dr. Cartwright produced a huge amount of material throughout her tenure delivering this course. Much of this material is still relevant to the course and is available at:
    
    \noindent
    \url{http://www.hep.shef.ac.uk/cartwright/phy306/index.html}
    
    \noindent
    I strongly recommend you utilise this material. Simply having the same concepts presented in two different styles can really help with understanding.
    
    \section{A Note on Capitalisation}
   Throughout the course, you'll see that I sometimes spell ``universe'' with a lower case u, and sometimes with an upper case U. This is not because I'm being sloppy and inconsistent with my grammar (well, I hope not most of the time!). Instead, it's because we'll consider both {\bf our} Universe (i.e., the one real Universe in which we reside) and {\bf model} universes. The first is a proper noun (and thus starts with a capital letter), whereas the second is a common noun. Think of it a bit like God and god. The first would be the deity one chooses to believe in (e.g., the one God of the Abrahamic religions), whereas the second refers to the general concept of god(s) (e.g., we speak today of the gods of ancient Egypt).
    
    \end{document}