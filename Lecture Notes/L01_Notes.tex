\documentclass[11pt]{article}
\renewcommand{\baselinestretch}{1.05}
\usepackage{amsmath,amsthm,verbatim,amssymb,amsfonts,amscd, graphicx}
\usepackage{graphics}

\topmargin0.0cm
\headheight0.0cm
\headsep0.0cm
\oddsidemargin0.0cm
\textheight23.0cm
\textwidth16.5cm
\footskip1.0cm
\newcommand{\inchsign}{$^{\prime\prime}$}

 \begin{document}
 
\title{Lecture 1:\\Fundamental Observations of the Universe}
\author{Dr. James Mullaney}
\maketitle

\section{Early history and evidence for a non-static Universe}
\begin{itemize}
\item Throughout history, various religions have offered their own explanation for what we'd now call the Universe.
\item Some of these involve a beginning to the Universe. For example, the ``In the beginning, God created heaven and the earth'' of the Bible.
\item However, another possibility is that the Universe has been around for ever.
\item And first impressions suggest that the Universe is static and non-evolving.
\item The first documented, logically-sound argument against a non-infinite (in time), evolving Universe is Olber's Paradox.
\end{itemize}

\subsection{Olber's paradox}
Olber's paradox concerns the question of why the sky is dark. At
first, it may seem obvious why the sky is dark. However, if the
Universe is infinite in both size and age (and contains a roughly
constant density of stars/galaxies), then there's a scientifically robust proof that the night sky should, in fact, be bright.
\vspace{3mm}

\noindent
{\bf Slide 4}
\begin{itemize}
  \item Imagine looking out into an infinite universe that is filled with stars at some average density (i.e., $n_\ast$ stars per unit volume).
  \item Eventually, because this universe is infinite, every line of sight will encounter a star.
  \item We can calculate the typical distance before our line of sight encounters a star - we'll denote this distance $\lambda$.
  \item To do this, imagine a cyclinder of length $\lambda$ and radius the same of that of a typical star $R_\ast$. The volume of this cylinder is:
  \begin{equation}
    V = \lambda\pi R_\ast^2
  \end{equation}
  \item The number of stars within this cyclinder is:
  \begin{equation}
    N_\ast = n_\ast\lambda\pi R_\ast^2
  \end{equation}
  \item To calculate the typical distance until our line of sight encouters one star, we set $N\ast=1$, giving:
  \begin{equation}
  \lambda = \frac{1}{n_\ast\pi R^2}
  \end{equation}
  \item In the current Universe, $n_\ast\sim10^9~{\rm Mpc^{-3}}$ and $R\ast$ is typically about $7\times10^8~{\rm m}$, giving $\lambda\sim10^{18}~{\rm Mpc}$.
  \item So our line of sight would typically extend to $10^{18}~{\rm Mpc}$ before encountering a star.
  \item This is much larger than the size of the observable Universe, but it is still a finite number.
  \item In an infinite (in age and size) universe, our line of sight {\it would} eventually extend to this kind of distance before hitting a star.
  \item If a star is at distance $\lambda$, then it's angular area relative to the area of the whole sky is:
  \begin{equation}
    \Omega_\ast = \frac{\pi R_\ast^2}{4\pi \lambda^2}
  \end{equation}
  and the flux we measure is:
  \begin{equation}
    f_\ast = \frac{L_\ast}{4\pi \lambda^2}
  \end{equation}
  where $L_\ast$ is the star's luminosity.
  \item Meaning its surface brightness is:
  \begin{equation}
  \Sigma = \frac{f}{\Omega}=\frac{L_\ast}{4\pi \lambda^2}\frac{4\lambda^2}{R_\ast^2} = \frac{L_\ast}{\pi R_2}
  \end{equation}
  \item The surface brightness is independent of $\lambda$, the distance to the star.
  \item Since every line of sight eventually hits a star, this means that in an infinite (in age and size) universe, the surface brightness of every part of the sky will be equal to the surface brightness of a typical star (such as our Sun)!
  \item There is no trickery here, meaning that our assumption of a static, infinite (in age and space) universe is incorrect.
  \item It turns out that the main reason for Olber's Paradox is that the real Universe is not infinite in age, meaning that the light from the most distant stars has not had time to reach us yet.
\end{itemize}

\section{Isotropy and homogeneity}
{\bf Slide 5}
\begin{itemize}
\item Another key observed property of the Universe is that it is {\it isotropic} and {\it homogeneous} on large scales.
\item ``Large scales'' refers to $>100~Mpc$.
\item Isotropic means ``no preferred direction''
\item Homogeneous means ``no preferred location''
\end{itemize}

\vspace{3mm}
\noindent
{\bf Slide 6}
\begin{itemize}
\item But this is clearly only the case on vary large scales.
\end{itemize}

\section{Distance is proportional to redshift}
{\bf Slide 7}
\begin{itemize}
\item When we look at galaxies beyond the local group (i.e., beyond the local effects of gravity), we see that they are all moving away from us.
\item We know this because the light from these galaxies is {\it redshifted}.
\item Recall, redshift is proportional to velocity - more highly redshifted galaxies are moving away from us at higher velocities.
\item We have also measured that more distant galaxies have higher redshifts (i.e., higher relative velocities, $v$), such that distance, $D$, is proportional to redshift, $z$. This is usually expressing in terms of Hubble's law: $v = H_0D$.
\end{itemize}

\vspace{3mm}
\noindent
{\bf Elastic example}
\begin{itemize}
  \item Imagine a one-dimensional universe - it only has length.
  \item Galaxies distributed along the universe.
  \item The location of the galaxies can be described in terms of co-moving coordinates.
  \item Because the expansion of the universe is isotropic -- the same in all directions -- the relative positions of the galaxies don't change.
  \item Their co-moving coordinates don't change.
  \item The real coordinates of every galaxy in the universe can be expressed as their co-moving coordinates multiplied by a scale factor $a(t)$ {\it that is the same for every galaxy}.
  \item Again, because it is isotropic and homogeneous, the expansion of the whole universe is {\it wholly} encapsulated in $a(t)$.
  \item As a reference point, we define $a(t)$ at the present moment, $t_0$, to be 1. (i.e., $a(t_0)=1$).
  \item In an expanding universe, $a(t)$ was $<1$ in the past, and will be $>1$ in the future. It would be the opposite in a contracting universe.
\end{itemize}

\vspace{3mm}
\noindent
{\bf Slide 9}
\begin{itemize}
\item Since the real coordinates of every galaxy can be expressed as its co-moving coordinates multiplies by the same scale factor. The real distance between any two galaxies at a given time, $D_{1,2}(t)$, can also be expressed as their distance in co-moving coordinates, $r_{1,2}$,  multiplied by the same scale factor, $a(t)$:
\begin{equation}
D_{3,1} = a(t)r_{3,1}
\end{equation}
\item And their relative velocities are given by:
\begin{equation}
  \label{v12}
  v_{1,2} = \frac{dD_{1,2}}{dt} = \frac{da(t)}{dt}r_{1,2} = \dot{a}r_{1,2}
\end{equation}
\item Substituting $r_{1,2} = D_{1,2}/a(t)$ into Eq. \ref{v12} gives: 
\begin{equation}
  v_{1,2} = \frac{\dot{a}}{a}D_{1,2}
\end{equation}
\item {\bf Slide 10} And comparing to $v_{1,2}=H_0D_{1,2}$ reveals that: 
\begin{equation}
  H_0 = \frac{\dot{a}}{a}
\end{equation}
evaluated at $t=t_0$, i.e., today.
\item But, this relationship holds true at any time, meaning that:
\begin{equation}
  H(t) = \frac{\dot{a(t)}}{a(t)}
\end{equation}
\item In other words, the Hubble Parameter is the ratio of the relative rate of change of the scale factor.
\item This is the first {\bf Important Equation} of the course.
\item Note: For historical reasons, $H_0$ is called the {\it Hubble Constant}, even though it's not a constant in time (but it is a constant in space)! $H(t)$ (or sometimes just $H$) is called the {\it Hubble Parameter}.  
\end{itemize}

\end{document}