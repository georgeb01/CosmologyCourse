\documentclass[11pt]{article}
\renewcommand{\baselinestretch}{1.05}
\usepackage{amsmath,amsthm,verbatim,amssymb,amsfonts,amscd, graphicx}
\usepackage{graphics}

\topmargin0.0cm
\headheight0.0cm
\headsep0.0cm
\oddsidemargin0.0cm
\textheight23.0cm
\textwidth16.5cm
\footskip1.0cm
\newcommand{\inchsign}{$^{\prime\prime}$}

 \begin{document}
 
\title{Lecture 2:\\The Shape of the Universe and Cosmological Distances}
\author{Dr. James Mullaney}
\maketitle

\section{Understanding Gravity}
{\bf Slide 2}
\begin{itemize}
\item The only signiciant force acting on large (i.e., Universe-sized) scales is gravity.
\item Therefore to understand how the Universe has evolved, we must first have a good understanding of gravity.
\item Newton was the first to attempt this. His laws are still a very good approximation for most circumstances.
\item In Newton's laws, the strength of gravitational force between two objects is dictated by a property known as their ``gravitational mass'', $m_g$.
\item This force acts on an object's ``inertial mass'', $m_i$, telling that object how to move.
\item It just so happens that, to within the errors of our best measurements, $m_g = m_i$. This is known as the ``equivalence principle''.
\item The equivalence principle implies that there is a unique acceleration due to gravity everywhere in the universe that is independent of $m$.
\item But Newton's theories didn't really tell us what gravity actually is.
\item {\bf Slide 3} Based on the Equivalence Principle, Einstein realised that there was no experiment that could tell the difference between constant acceleration and gravity.
\item {\bf Slide 4} This means that a beam of light behaves the same within a gravitational field as it does within a rocket undergoing constant acceleration.
\item The beam of light will appear to bend.
\item {\bf Slide 5} But Fermat's theorem states that light travels via the shortest possible route, which Einstein knew as via straight lines in {\it spacetime}.
\item So if the presence of mass causes light to {\it appear} to travel through a curve, but is actually travelling in straight lines in spacetime, it must mean that {\it spacetime is curved} by the presence of mass.
\item Therefore, spacetime {\it must be curved}.

\end{itemize}

\section{Describing curvature}
\begin{itemize}
\item On realising that spacetime -- the very fabric of the Universe -- was curved, the few early cosmologists started investigating means of describing curvature.
\item For our purposes, this is usually done in terms of measuring distances between points on a curved (or flat) surface.
\item Isotropic, homogeneous objects (i.e., like the Universe) can only have three different types of curvature (and one of them isn't even curved):
\begin{itemize}
\item Flat
\item Positively curved everywhere - like a sphere.
\item Negativelty curved everywhere - like a saddle-shape.
\end{itemize}
\item {\bf Slide 6} shows how to calculate the distance between two points on each of these different curved 2D surfaces.
\item But for curved spacetime, we need to consider distances between ``events'' in four dimensions.
\item {\bf Slide 7} The equation for calculating distances between two events is known as the {\bf Robertson-Walker metric}.
\end{itemize}

\section{What is ``distance''?}
\begin{itemize}
\item Before going further, we should define what we mean by ``distance'' within an expanding or contracting Universe, especially when the messenger -- light -- can take a very long time to reach us.
\item The first -- and most important -- distance we'll consider is the {\it Proper Distance}.
\item Proper distance is the distance you would measure with a hypothetical, infintely long, inflexible ruler.
\item As we'll see later in the course, there are other types of distance that differ from proper distance.
\item {\bf Slide 8} The {\it current} Proper Distance to a galaxy is the distance to that galaxy {\it right now, at this instant}. This is {\it not} the distance at which the galaxy {\it ``appears''} to be since, in an expanding universe, the galaxy has moved further away in the intervening period between when the light was emitted and the time when we observe it.
\item {\bf Slide 9} The current proper distance to a galaxy is the integral of the Robertson-Walker metric.
\item Because it is at {\it this instance}, then $dt=0$, and because we're measuring distance along the radial direction $d\theta=0$ and $d\phi=0$.
\item The RW metric then reduces to:
\begin{equation}
ds = a(t)dr
\end{equation}
\item And integrating over $r$ gives the proper distance:
\begin{equation}
d_p = a(t)\int_0^r dr = a(t)r
\end{equation}
\end{itemize}

\section{Redshifts and distances}
\begin{itemize}
\item On cosmological scales, pretty much the only things we can measure for a galaxy are its flux (maybe in different filters), angular size, and redshift.
\item This would present serious difficulties for measuring cosmological distances if it weren't for the fact that redshift is directly related to the scale factor.
\item An interesting property of light is that it travels along the null geodesic. In other words, it travels along lines of {\it constant spacetime}.
\item This means that, for light, $ds = 0$.
\item Therefore, for light travelling along the radial direction (i.e., $d\theta=0$ and $d\phi=0$), the RW metrtic reduces to:
\begin{equation}
    0 = -c^2dt^2+a(t)^2dr^2
\end{equation}
or
\begin{equation}
    \label{cdt}
    \frac{cdt}{a(t)}=dr
\end{equation}
\item To see how this can be related to redshift, we'll consider a single photon emitted by a galaxy at time $t_{\rm em}$ and observed at time $t_{\rm ob}$.
\item Integrating both sides of Eq. \ref{cdt} between $t_{\rm em}$ and $t_{\rm ob}$ gives:
\begin{equation}
    \label{cint1}
    c\int_{t_{\rm em}}^{t_{\rm ob}}\frac{dt}{a(t)}=\int_0^r dr=r
\end{equation}
since in the intervening time, the light has travelled across the distance $r$ in comoving coordinates.
\item The next crest of the photon's wavelength is emitted at $t_{\rm em}+\lambda_{\rm em}/c$ and observed at $t_{\rm ob}+\lambda_{\rm ob}/c$, giving:
\begin{equation}
    \label{cint2}
    c\int_{t_{\rm em}+\lambda_{\rm em}/c}^{t_{\rm ob}+\lambda_{\rm ob}/c}\frac{dt}{a(t)}=\int_0^r dr=r
\end{equation}
\item Since the RHS of both Eqs. \ref{cint1} and \ref{cint2} are the same, we can say:
\begin{equation}
    \label{intt}
    \int_{t_{\rm em}+\lambda_{\rm em}/c}^{t_{\rm ob}+\lambda_{\rm ob}/c}\frac{dt}{a(t)}=\int_{t_{\rm em}}^{t_{\rm ob}}\frac{cdt}{a(t)}
\end{equation}
where the $c$s have been cancelled. 
\item From both sides, we then subtract:
\begin{equation}
    \int_{t_{\rm em}+\lambda_{\rm em}/c}^{t_{\rm ob}}\frac{dt}{a(t)}
\end{equation}
which, for the LHS of Eq. \ref{intt}, corresponds to:
\begin{equation}
    \left.\int\frac{dt}{a(t)} \right|_{t_{\rm ob}+\lambda_{\rm ob}/c} - \left.\int\frac{dt}{a(t)} \right|_{t_{\rm em}+\lambda_{\rm em}/c} - 
    \left.\int\frac{dt}{a(t)} \right|_{t_{\rm ob}} +
    \left.\int\frac{dt}{a(t)} \right|_{t_{\rm em}+\lambda_{\rm em}/c}
\end{equation}
with the second and fourth term cancelling.
\item Doing the same for the RHS of Eq. \ref{intt}, it then becomes:
\begin{equation}
    \int^{t_{\rm em}+\lambda_{\rm em}/c}_{t_{\rm em}}\frac{dt}{a(t)}=
    \int_{t_{\rm ob}}^{t_{\rm ob}+\lambda_{\rm ob}/c}\frac{dt}{a(t)}
\end{equation}    

\item Over the time between the emission of the first and the second crest of the photon's wave, $a(t)$ is effectively constant, so can come out of both integrals:
\begin{equation}
    \frac{1}{a(t_{\rm em})}\int^{t_{\rm em}+\lambda_{\rm em}/c}_{t_{\rm em}}dt=\frac{1}{a(t_{\rm ob})}\int_{t_{\rm ob}}^{t_{\rm ob}+\lambda_{\rm ob}/c}dt
\end{equation}

\item Performing the integral then gives:
\begin{equation}
    \frac{1}{a(t_{\rm em})}\left[t_{\rm em}+\lambda_{\rm em}/c - t_{\rm em}\right]=\frac{1}{a(t_{\rm ob})}\left[t_{\rm ob}+\lambda_{\rm ob}/c - t_{\rm ob}\right]
\end{equation}


\item Cancelling $t_{\rm em}$, $t_{\rm ob}$, and $c$ gives:  
\begin{equation}
    \frac{\lambda_{\rm em}}{a(t_{\rm em})} = \frac{\lambda_{\rm ob}}{a(t_{\rm ob})}
\end{equation}
\item or
\begin{equation}
    \frac{a(t_{\rm ob})}{a(t_{\rm em})} = \frac{\lambda_{\rm ob}}{\lambda_{\rm em}}
\end{equation}
\item and using 
\begin{equation}
    z = \frac{\lambda_{\rm ob}-\lambda_{\rm em}}{\lambda_{\rm em}}
\end{equation}
gives
\begin{equation}
    \frac{a(t_{\rm ob})}{a(t_{\rm em})} = \frac{1}{a(t_{\rm em})} = 1+z
\end{equation}
\item Meaning that by measuring the redshift of a galaxy immediately tells us the relative scale factor of the Universe when the light from that galaxy was emitted.
\item This is the second {\bf Important Equation} of the course.
\end{itemize}
\end{document}
