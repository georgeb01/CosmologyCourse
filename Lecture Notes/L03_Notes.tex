\documentclass[11pt]{article}
\renewcommand{\baselinestretch}{1.05}
\usepackage{amsmath,amsthm,verbatim,amssymb,amsfonts,amscd, graphicx}
\usepackage{graphics}

\topmargin0.0cm
\headheight0.0cm
\headsep0.0cm
\oddsidemargin0.0cm
\textheight23.0cm
\textwidth16.5cm
\footskip1.0cm
\newcommand{\inchsign}{$^{\prime\prime}$}

 \begin{document}
 
\title{Lecture 3:\\The Friedmann Equation}
\author{Dr. James Mullaney}
\maketitle

\section{The Robertson Walker metric}
{\bf Slide 2}
\begin{itemize}
\item Saw in lecture 2 that an isotropic, homogeneous Universe can be fully described by just three numbers: $\kappa$, $R_0$, $a(t)$.
\item $a(t)$ is paricularly important: it tells us how the Universe expands and contracts over time.
\item $a(t)$ also enables us to relate redshifts (which are easily measured) to distances (which are much more difficult to measure).
\item However, let's first focus on curvature, described by $\kappa$ and $R_0$.
\end{itemize}
\subsection{What is curvature in the context of a universe?}
{\bf Slide 3}
\begin{itemize}
\item Curvature in a universe would manifest itself in terms of the perceived sizes of distant objects.
\item The {\it perceived} size of an object is the angular size it subtends.
\item As such, it is convenient to think in terms of a triangle, with the object at one end, and the angular size at the other.
\item In a flat universe, the angles of this triangle all add up to 180 degrees, and the objects subtends the angle that we have come to expect for flat geometry.
\item In a negatively curved universe, the interior angles of the triangle add up to less than 180 degrees, and the object subtends a {\it smaller} angle that we would expect. This would be witnessed as distant galaxies appearing disproportionately smaller than nearby ones.
\item By contrast, in a negatively curved universe, the interior angles of the triangle add up to more than 180 degrees, and the object subtends a {\it larger} angle that we would expect. This would be witnessed as distant galaxies appearing disproportionately larger than nearby ones.
\item Such disproportionately large or small distant galaxies are not seen when we look to higher and higher distances (i.e., redshifts), so we can conclude that {\it if} the Universe is curved, then its radius of curvature, $R_0$ is much larger than the size of the observable Universe.

\end{itemize}

\section{Relating curvature to content}
\begin{itemize}
\item General relativity tells us that a universe's curvature is dictated by its content (whether mass or energy, since they are one and the same thing in relativity).
\item The {\it Field Equation} links the two - it tells spacetime ($G_{\mu,\nu}$) how to curve in the presence of stress-energy ($T_{\mu, \nu}$).
\item Unfortunately, both $G_{\mu,\nu}$ and $T_{\mu, \nu}$ are $4\times4$ tensors (i.e., matrices), and the equation as a whole represents ten non-linear second-order differential equations!
\item Thus, in general, it can be extremely difficult to solve for $G_{\mu,\nu}$.
{\bf Slide 5}
\item However, on large scales, we can make some sweeping simplifications.
\item On very large scales, we can ignore the ``clumpy'' nature of the Universe and instead describe it as being filled with a uniform (i.e., homogeneous), isotropic gas or pressure $P(t)$ and (energy) density $\varepsilon(t)$.
\item In such a case, then $T_{\mu,\nu}$ only depents on $P(t)$ and $\varepsilon(t)$, and the metric is given by the Robertson Walker metric.
\item Our goal, therefore, is simply to relate $a(t)$, $\kappa$, and $R_0$ to $P(t)$ and $\varepsilon(t)$.
\end{itemize}

\section{The Friedmann Equation}

\end{document}