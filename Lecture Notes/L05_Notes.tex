\documentclass[11pt]{article}
\renewcommand{\baselinestretch}{1.05}
\usepackage{amsmath,amsthm,verbatim,amssymb,amsfonts,amscd, graphicx}
\usepackage{graphics}

\topmargin0.0cm
\headheight0.0cm
\headsep0.0cm
\oddsidemargin0.0cm
\textheight23.0cm
\textwidth16.5cm
\footskip1.0cm
\newcommand{\inchsign}{$^{\prime\prime}$}
\newcommand{\vep}[1]{\ensuremath{\varepsilon#1}}

\begin{document}
 
\title{Lecture 05:\\Solving the Friedmann Equation {\sc II}:\\
Our first model universe}
\author{Dr. James Mullaney}
\maketitle
\section{Three equations}
\begin{itemize}
\item The Friedmann Equation (F.E.) gives us the means to determine how, given a set energy densities, the scale factor changes over the history (and future) of the universe:
\begin{equation}
H^2 = \left(\frac{\dot{a}}{a}\right)^2 = \frac{8\pi G}{3c^2}\vep{(t)} - \frac{\kappa c^2}{R_0}\frac{1}{a(t)}
\end{equation}
\item Since the scale factor quite literally describes the expansion and contraction of the universe, the F.E. provides the means to work out the dynamic evolution of the universe.
\item What we now need to do is solve the F.E. to obtain an expression for $a(t)$ for a given set of energy densities.
\item However, in its standalone form, the F.E. is not enough to work out how the universe expands or contracts.
\item We also need expressions for the various energy densities, \vep{(t)} (for matter, radiation and dark energy) which, from the First Law of Thermodynamics, are given by:
\begin{equation}
\dot{\vep{}} + 3\frac{\dot{a}}{a}(\vep{} + P) = 0
\end{equation}
\item While the pressure, $P$, is given by the Equation of State:
\begin{equation}
        P = \vep{}\omega
\end{equation}
\item Of course, things are slightly complicated by the fact that we've got more than one type of energy density.
\item Thankfully, however, the energy densities can be added linearly, meaning the total energy density is given by:
\begin{equation}
\vep{_{\rm Tot}} = \sum_i\vep{_{i}} = \vep{_{\rm m}} + \vep{_{\rm p}} + \vep{_{\rm d}}
\end{equation} 
and the total pressure is given by:
\begin{equation}
    P_{\rm Tot} = \sum_i P_{i} = \sum_i\vep{_i}\omega_i = \vep{_{\rm m}}\omega_{\rm m} + \vep{_{\rm p}}\omega_{\rm p} + \vep{_{\rm d}}\omega_{\rm d}
\end{equation} 
\item As such, the Fluid Equation must hold for all three types of energy density separately:
\begin{equation}
\dot{\vep{_i}} + 3\frac{\dot{a}}{a}(\vep{_i} + P_i) = 0
\end{equation}
\end{itemize}

\section{The evolving energy densities}
\begin{itemize}
\item To solve the F.E. we require expressions for \vep{(t)} for the various types of energy density.
\item But, what would be {\it even more useful} would be to have equivalent espressions for \vep{} in terms of scale factor, a. In other words \vep{(a)}.
\item This would tell us how energy density changes as the universe expands or contracts.
\item For this, we can write the Fluid Equation as:
\begin{equation}
\frac{d\vep{_i}}{dt} + 3\frac{da}{dt}\frac{1}{a}\vep{_i}(1 + \omega_i) = 0
\end{equation}
\item Cancellling the $dt$s and dividing both sides by \vep{} gives:
\begin{equation}    
\frac{d\vep{_i}}{\vep{_i}} + 3\frac{da}{a}(1 + \omega_i) = 0
\end{equation}
\item Since $\omega$ is independent of time and scale factor, integrating then gives:
\begin{equation}
{\rm ln}(\vep{_i}) = -3(1+\omega_i){\rm ln}(a) + b
\end{equation}
where $b$ is a constant of integration. Equivalently:
\begin{equation}
    \vep{_i(a)} = Ba^{-3(1+\omega_i)}
\end{equation}
where $B=e^b$.
\item To determine $B$, we recall that the current scale factor is defined to be $a(t_0) = 1$, and the current energy density is \vep{_{i,0}}, so:
\begin{equation}
    \vep{_{i,0}} = B\times1^{-3(1+\omega_i)}
\end{equation}
meaning $B=\vep{_{i,0}}$, and
\begin{equation}
    \vep{_{i}(a)} = \vep{_{i,0}}a^{-3(1+\omega_i)}
\end{equation}
\item Substituting the values for $\omega$ given in the previous lecture gives:
\begin{itemize}
\item Matter: $\omega = 0$, giving $\vep{_{\rm m}} = \vep{_{\rm m,0}}a^{-3} = \vep{_{\rm m,0}}/{a^3}$ 
\item Radiation: $\omega = 1/3$, giving $\vep{_{\rm p}} = \vep{_{\rm p,0}}a^{-4} = \vep{_{\rm p,0}}/{a^4}$
\item Dark Energy: $\omega = -1$, giving $\vep{_{\rm d}} = \vep{_{\rm d,0}}a^{0} = \vep{_{\rm m,0}}$
\end{itemize}
\item The first (i.e., matter) makes sense: as the universe expands, matter gets diluted as the volume of the universe increases.
\item The second (i.e., radiation) at first doesn't makes sense. Why would the energy density of photons decrease faster that that due to volume dilution? It's because as well as volume dilution, the wavelengths of the photons expand as the universe expands, meaning each indivual photon's energy also falls as the scale factor increases (due to $E=hf$).
\item But the last one (i.e., Dark Energy) makes least sense of all -- the energy density of Dark Energy is constant with respect to the scale factor. This means that the Dark Energy density does not get diluted as the universe expands. Each ``new'' cubic meter of the universe is ``produced'' with its own allocation of Dark Energy!
\end{itemize}
\end{document}