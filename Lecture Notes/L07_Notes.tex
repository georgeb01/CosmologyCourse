\documentclass[11pt]{article}
\renewcommand{\baselinestretch}{1.05}
\usepackage{amsmath,amsthm,verbatim,amssymb,amsfonts,amscd, graphicx}
\usepackage{graphics}

\topmargin0.0cm
\headheight0.0cm
\headsep0.0cm
\oddsidemargin0.0cm
\textheight23.0cm
\textwidth16.5cm
\footskip1.0cm
\newcommand{\inchsign}{$^{\prime\prime}$}
\newcommand{\vep}[1]{\ensuremath{\varepsilon#1}}

 \begin{document}
 
\title{Lecture 07:\\The ``Benchmark Model'' and \\``measurable'' distances}
\author{Dr. James Mullaney}
\maketitle
\section{The Benchmark Model for the real Universe}
\begin{itemize}
\item Up to this point, we've only considered ``simplistic'' universes which are either empty, or which just consist of a single component (e.g., matter {\it or} radiation, etc.).
\item However, we know that the real Universe contains (at least) three main components: radiation, matter and Dark Energy (I say ``at least'', since there may be other components that are too insignificant to currently measure).
\item Thankfully, things are made easier by different components adding linearly in the Friedmann Equation, i.e.,:
\begin{equation}
    \varepsilon{_{\rm tot}} = \varepsilon{_{\rm R}}+\varepsilon{_{\rm M}}+\varepsilon{_{\rm D}}
\end{equation}
\item From this, and by using the relationships derived in previous lectures:
\begin{align}
\varepsilon{_0,c}&=\frac{3H_0^2c^2}{8\pi G}\\
\Omega_{i,0}&=\frac{\varepsilon{_i,0}}{\varepsilon{_c,0}}\\
H_0^2(1-\Omega_0)&= \frac{-\kappa c^2}{R_0^2}\\
\varepsilon{(a)}&=\varepsilon{_0}a^{-3(1+\omega)}\\
\end{align}
we get:
\begin{equation}
\left(\frac{H}{H_0}\right)^2=\Omega
\end{equation}
\end{itemize}

\end{document}