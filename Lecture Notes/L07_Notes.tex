\documentclass[11pt]{article}
\renewcommand{\baselinestretch}{1.05}
\usepackage{amsmath,amsthm,verbatim,amssymb,amsfonts,amscd, graphicx}
\usepackage{graphics}
\usepackage{bigints}

\topmargin0.0cm
\headheight0.0cm
\headsep0.0cm
\oddsidemargin0.0cm
\textheight23.0cm
\textwidth16.5cm
\footskip1.0cm
\newcommand{\inchsign}{$^{\prime\prime}$}
\newcommand{\vep}[1]{\ensuremath{\varepsilon#1}}

 \begin{document}
 
\title{Lecture 07:\\The ``Benchmark Model'' and \\``measurable'' distances}
\author{Dr. James Mullaney}
\maketitle
\section{The Benchmark Model for the real Universe}
\begin{itemize}
\item Up to this point, we've only considered ``simplistic'' universes which are either empty, or which just consist of a single component (e.g., matter {\it or} radiation, etc.).
\item However, we know that the real Universe contains (at least) three main components: radiation, matter and Dark Energy (I say ``at least'', since there may be other components that are too insignificant to currently measure).
\item Thankfully, things are made easier by different components adding linearly in the Friedmann Equation, i.e.,:
\begin{equation}
    \varepsilon{_{\rm tot}} = \varepsilon{_{\rm R}}+\varepsilon{_{\rm M}}+\varepsilon{_{\rm D}}
\end{equation}

\item Which means the F.E. for the real Universe becomes:
\begin{equation}
    \left(\frac{\dot{a}}{a}\right)^2 = \frac{8\pi G}{3c^2}\sum_i\varepsilon{_i}-\frac{\kappa c^2}{R_0^2a(t)^2}
\end{equation}
    

\item From this, and by using the relationships derived in previous lectures:
\begin{align}
    H &= \left(\frac{\dot{a}}{a}\right)\\
    \Omega_{i,0}&=\frac{\varepsilon{_{i,0}}}{\varepsilon{_{c,0}}}\\
    \varepsilon{_{0,c}}&=\frac{3H_0^2c^2}{8\pi G}\\
    H_0^2(1-\Omega_0)&= \frac{-\kappa c^2}{R_0^2}\\
    \varepsilon{(a)_i}&=\varepsilon{_{i,0}}a^{-3(1+\omega)}
\end{align}
we get:
\begin{equation}
\left(\frac{H}{H_0}\right)^2=\frac{\Omega_{\rm p, 0}}{a^4}+\frac{\Omega_{\rm m, 0}}{a^3} + \Omega_{\rm d, 0} + \frac{1-\Omega_0}{a^2}
\end{equation}
\item meaning:
\begin{equation}
    \frac{da}{dt}=H_0\left(\frac{\Omega_{\rm p, 0}}{a^2}+\frac{\Omega_{\rm m, 0}}{a} + \Omega_{\rm d, 0}a^2 + 1-\Omega_0\right)^{1/2}
\end{equation}
\item So, to obtain $a(t)$, we need to solve the following:
\begin{equation}
        \bigints_0^a{\frac{da}{\left(\frac{\Omega_{\rm p, 0}}{a^2}+\frac{\Omega_{\rm m, 0}}{a} + \Omega_{\rm d, 0}a^2 + 1-\Omega_0\right)^{1/2}}} = H_0t
\end{equation}
\item For the real Universe:
\begin{align}
\Omega_{p,0}&=9\times10^{-5}\\
\Omega_{m,0}&=0.31\\
\Omega_{d,0}&=0.69\\
\Omega_{0}&=1.00\\
H_0 &= 67.7~{\rm km~s^{-1}~Mpc}
\end{align}
\item Unfortunately, this cannot be solved analytically, so instead we solve it numerically using computers to integrate ``under the curve''.
\item See notes on lecture slides...
\end{itemize}

\section{Luminosity Distance}
\begin{equation}
    d_L = \left(\frac{L}{4\pi F}\right)^{1/2}
\end{equation}

\begin{equation}
d_A = d_p(t_{\rm em}) = \frac{d_p(t_{\rm ob})}{1+z}
\end{equation}

\end{document}