\documentclass[11pt]{article}
\renewcommand{\baselinestretch}{1.05}
\usepackage{amsmath,amsthm,verbatim,amssymb,amsfonts,amscd, graphicx}
\usepackage{graphics}

\topmargin0.0cm
\headheight0.0cm
\headsep0.0cm
\oddsidemargin0.0cm
\textheight23.0cm
\textwidth16.5cm
\footskip1.0cm
\newcommand{\inchsign}{$^{\prime\prime}$}
\newcommand{\vep}[1]{\ensuremath{\varepsilon#1}}

 \begin{document}
 
\title{Lecture 09:\\The radiation content of the Universe:\\The Cosmic Microwave Background Part 1}
\author{Dr. James Mullaney}
\maketitle

\section{The radiation content of the Universe}
\begin{itemize}
\item Galaxy surveys tell us that the density of stars in the Universe corresponds to about $1.7~{\rm L\odot~Mpc^{-3}}$.
\item If, as an upper limit, we assume that these stars have emitted photons since the beginning of the Universe, $4.5\times10^{17}~{\rm s}$ ago, it corresponds to an energy density arising from stellar photons of: $10^{15}~{\rm J m^{-3}}$.
\item The other main source of photons in the Universe is the CMB.
\item The spectrum of the CMB is a black body of $T=2.7~K$, so we can calculate its energy density using:
\begin{equation}
E = \alpha T^4
\end{equation}
where
\begin{equation}
\alpha=\frac{\pi^2}{15}\frac{k^4}{\hbar^3 c^3}=7.566\times10^{-16}~{\rm J~m^{-3}~K^{-4}}=0.2606~{\rm MeV~m^{-3}}
\end{equation}
and $k$ is the Boltzmann constant.
\item So the energy density of the CMB is {\it at least} a factor of 40 times that of stellar photons.
\item Despite this, the energy density of the CMB still represents only a small fraction of the critical density of our Universe (i.e., $\Omega_{\rm p} = 5.35\times10^{-5}$).
\item However, because the energy of each CMB photon is very low, $hf_{\rm mean}=6.34\times10^{-4}~{\rm eV}$, the number density of CMB photons (i.e., the number of photons per unit volume) is very high:
\begin{equation}
n_{\rm p} = \frac{0.2606\times10^6}{6.34\times10^{-4}}=4.107\times10^8~{\rm m^{-3}}
\end{equation}
\item Indeed, compared to the number density of Baryons:
\begin{equation}
    n_{\rm b} = \frac{\Omega_{\rm b,0}\vep{_{c,0}}}{E_{\rm b}}=\frac{0.048\times4890~{\rm MeV~m^{-3}}}{939~{\rm MeV}} = 0.25~{\rm m^{-3}}
\end{equation}
\end{itemize}

\section{Recombination}
\begin{itemize}
\item In an expanding universe, such as our own, as we go further back in time the scale factor, $a$, gets smaller and smaller.
\item This, in turn, means that the wavelength of CMB photons get shorter and shorter, meaning their energies get higher and higher.
\item Since the mean energy, $E$, of a CMB photon is related to temperature, $T$, via $E=2.7kT$ (where $k$ is the Boltzman constant), we can calculate the temperature of the Universe as a function of $a$:
\begin{equation}
    hf_{\rm mean}=\frac{hc}{\lambda_{\rm mean}}=\frac{hc}{a\lambda_0} = 2.7kT(a)
\end{equation}
or, equivalently, $T(a) = T_0/a$, where $T_0$ is the temperature of the CMB today (i.e., 2.755\ K).
\item If we go far enough back in time, then the energies of a large proportion of CMB photons exceed that of the ionisation energy of Hydrogen.
\item And since there are 1.6 billion CMB photons for every Baryon, the instant an electron binds to a proton to form a Hydrogen atom, it gets blasted apart by an ionising photon.
\item However, as the Universe expands, the photons and electrons move further away from each other, so they interact less.
\item If nothing were to change, the distance between the photons and electrons would become so large that they would eventually stop interacting.
\item Before that happens, however, we get recombination.
\item As the Universe expands, the energy of the CMB photons drops as their wavelengths increase.
\item Recombination occurs when there are insufficient numbers of photons with energy greater than the ionisation energy of Hydrogen to keep it ionised.
\item When does this occur?
\item One guess would be when the mean energy per photon drops below the ionisation energy of Hydrogen. With the mean photon energy given by $2.7kT$, this corresponds to:
\begin{equation}
T_{\rm rec} = \frac{13.6}{2.7k}=\frac{13.6}{2.7\times8.6\times10^{-5}}\sim60,000~{\rm K}
\end{equation}
which occured when $a=2.7/60000=4.5\times10^{-5}$, or $z\sim22,000$.
\item However, it turns out this is a poor guess.
\item The reason being that, with 1.6 billion photons per baryon, the Universe doesn't need roughly {\it half} the CMB to be able to ionise H, it needs a much smaller fraction.
\item Instead, we can use the Saha Equation to determine the fraction of ionised atoms:
\begin{equation}
\label{nh1}
    \frac{n_H}{n_pn_e} = \left(\frac{m_ekT}{2\pi\hbar^2}\right)^{-3/2}{\rm exp}\left(\frac{Q}{kT}\right)
\end{equation}
\item Firstly, we'll simplify things by assuming that the gas is 100\%\ Hydrogen.
\item We'll then define a quantity $X$ as the fraction of all the gas that is ionised:
\begin{equation}
    \label{nh2}
    X = \frac{n_p}{n_p + n_H}
\end{equation}
\item Which we can rearrange to get:
\begin{equation}
    n_H = \frac{1-X}{X}n_p
\end{equation}
\item Using the fact that $n_e=n_p$, and subbing Eq. \ref{nh1} into Eq. \ref{nh2} we get:
\begin{equation}
    \label{1mx}
    \frac{1-X}{X} = n_p\left(\frac{m_ekT}{2\pi\hbar^2}\right)^{-3/2}{\rm exp}\left(\frac{Q}{kT}\right)
\end{equation}
\item But we need to relate this to the photons that are maintaining the gas at temperature, $T$.
\item We do this by resorting to the fact that we know what the ratio of the number of baryons to photons is (was and will be), as highlighted at the start of the lecture. We'll call this ratio $\eta$:
\begin{equation}
    \label{eta}
    \eta = \frac{n_{\rm bary}}{n_\gamma} = \frac{n_p+n_H}{n_\gamma} = \frac{n_p}{Xn_\gamma}
\end{equation}
\item And that, since we're dealing with a black body, we {\it know} the number density of photons by integrating over the black body equation to give:
\begin{equation}
    n_\gamma = 0.2436\left(\frac{k T}{\hbar c}\right)^3
\end{equation}
\item which we can combine with Eq. \ref{eta} to give:
\begin{equation}
    \label{np}
    n_p = 0.2436X\eta\left(\frac{k T}{\hbar c}\right)^3
\end{equation}
\item This gives the number density of ionised Hydrogen atoms ($n_p$) given the ratio of baryons to photons ($\eta$, which we know), the temperature of the black body, $T$ (which we're trying to obtain), and the ratio of ionised to neutral gas $X$ (which we'll define as $1/2$ for the moment of recombination).
\item In other words, we've replaced $n_p+n_H$ with an expression that comes solely from considering the photons follow a black body distribution.
\item Inserting the expression for $n_p$ (Eq. \ref{np}) into Eq. \ref{1mx} and using the known quantities for some of the constants, we get:
\begin{equation}
\frac{1-X}{X^2} = 3.84\eta\left(\frac{kT}{m_ec^2}\right)^{3/2}{\rm exp}\left(\frac{Q}{kT}\right)
\end{equation}
\item Since $T$ is the only unknown in this equation, it can be solved. The easiest way to do this, however, is graphically (see lecture slides).

\end{itemize}

\end{document}