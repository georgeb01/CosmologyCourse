\documentclass[11pt]{article}
\renewcommand{\baselinestretch}{1.05}
\usepackage{amsmath,amsthm,verbatim,amssymb,amsfonts,amscd, graphicx}
\usepackage{graphics}

\topmargin0.0cm
\headheight0.0cm
\headsep0.0cm
\oddsidemargin0.0cm
\textheight23.0cm
\textwidth16.5cm
\footskip1.0cm
\newcommand{\inchsign}{$^{\prime\prime}$}
\newcommand{\vep}[1]{\ensuremath{\varepsilon#1}}

 \begin{document}
 
\title{Problems Class \sc{i}}
\author{Dr. James Mullaney}
\maketitle

\begin{enumerate}
\item Show that the observed surface brightness of a galaxy -- the flux per unit solid angle -- falls with distance as:
\begin{equation*}
S = \frac{L}{4\pi d_p(t_0)^2}\frac{1}{(1+z)^4}
\end{equation*}
where $L$ is luminosity of all the stars contained within a square kiloparcsec of the galaxy.

\item If my telescope can measure flux to within an accuracy of 10\%, at what redshifts do I have to worry about which cosmological distance I use when calculating the luminosity of a galaxy?
 
\item Derive the following form of the Friedmann Equation:
\begin{equation}
    \label{dota}
    \dot{a}^2 = H_0^2\left(\frac{\Omega_{\rm r,0}}{a^2}+\frac{\Omega_{\rm m,0}}{a}+\Omega_{\rm \kappa,0}+\Omega_{\rm D,0}a^2\right)
\end{equation}
\begin{enumerate}
\item From Eq. \ref{dota}, show that a universe in which $\kappa=+1$ and $\Omega_{\rm D,0}=0$ will reach a maximum value of $a$ before recollapsing.
\item What would the recollapse mean in terms of motions of bodies on sub-cosmological scales (i.e., galaxy groups, galaxies, planetary systems, you, me). Justify your answer.
\item By differentiating Eq. \ref{dota} in the case of where the universe is flat and the radiation density is negligible (i.e., a reasonable description of today's Universe), find the condition under which the expansion of the universe will currently be accelerating.
\end{enumerate}
\item The metal-poor star HE 1523-0901 has an age of $13.4\pm2.2~{\rm Gyr}$ based on uranium radiochemical dating (Frebel et al., 2007, {\it ApJ}, 660, L117). Assuming a matter-only model, calculate the value of $H_0$ that this would imply, if the first stars formed 100~Myr after the Big Bang.
\item In the 1950s it was generally-accepted that $H_0\sim 500~{\rm km~s^{-1}}$, with a random error estimated to be around 10-20\%.
\begin{enumerate}
\item Why do you think this early measurement differed so much from today's generally-accepted value of the Hubble constant, $H_0\approx70~{\rm km~s^{-1}}$?
\item Why did the generally-accepted value of $H_0$ in 1950 lead most astronomers to believe in the steady-state model of the Universe?
\end{enumerate}

\end{enumerate}
\end{document}