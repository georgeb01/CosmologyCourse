\documentclass[11pt]{article}
\renewcommand{\baselinestretch}{1.05}
\usepackage{amsmath,amsthm,verbatim,amssymb,amsfonts,amscd, graphicx}
\usepackage{graphics}

\topmargin0.0cm
\headheight0.0cm
\headsep0.0cm
\oddsidemargin0.0cm
\textheight23.0cm
\textwidth16.5cm
\footskip1.0cm
\newcommand{\inchsign}{$^{\prime\prime}$}
\newcommand{\vep}[1]{\ensuremath{\varepsilon#1}}

 \begin{document}
 
\title{Reading Exercise\\Based on the review article\\
{\it The First Galaxies}\\ by Broom \& Yoshida, 2011}
\maketitle
 
\noindent
The aim of this exercise is for you conduct self-directed study to gain insights into related material {\it beyond} the core material of the course. You are required to read a review article about the formation of the first galaxies soon after the Big Bang and answer the questions that follow. In doing so, you will consider early galaxy formation in more detail that will be covered in the lectures.

\vspace{2mm}
\noindent
The paper you are required to read is {\it The First Galaxies} by Broom \& Yoshida, 2011, ARAA, 49:373-407. It is available on course's MOLE page, as I am unable to make it world-viewable on GitHub due to copyright.

\vspace{2mm}
\noindent
Some of the questions will require to do a small amount of reading beyond the main review article. The numbers in brackets are meant as a guide to the number of marks that can be obtained from each question.

\vspace{2mm}
\noindent
Please submit your answers to F10 (including a cover sheet) by 5pm on Friday, 4th May.

\begin{enumerate}
\item The first paragraph of the Introduction refers to the ``cosmic dark ages''. Describe what is meant by this term. (1)
\item The review article refers a lot to ``Population~{\sc iii}'' (Pop {\sc iii}) stars. What is the accepted definition of Population {\sc iii} stars, and how do they differ from Population~{\sc ii} and Population~{\sc i} stars? (2)
\item What is meant by the term ``stellar feedback'' in the context of the review article. What are the two main forms of stellar feedback that the review article refers to? (4)
\item Provide an account of how the first stars affect further subsequent star-formation (3).
\item When discussing Black Holes (BHs) in the Introduction, the article states ``such massive BHs would likely have influenced the structure and evolution of the first galaxies''. Briefly explain {\it how} early BHs are thought to influence the first galaxies. You may wish to refer to the cited article, which is freely available on arXiv. (3)
\item What is meant by ``top-heavy'' and ``bottom-heavy'' in the context of star-formation? (2)
\item Why does a model in which the ``halos that host the formation of the first Pop {\sc iii} stars coincide with the first galaxies'' involve the implicit assumption that the initial mass function was not very different from today? (2)
\item Summarise the challenges in forming the first galaxies associated with the assumption that Pop {\sc iii} were predominantly massive. (4)
\item Throughout, the review article refers to a virial temperature of $10^4~{\rm K}$. What is the significance of this temperature? (2)
\item Considering Fig. 1 and its caption, why are the first galaxies unlikely to contain Pop {\sc iii} stars? (2)
\item What is the argument that the review provides for the possible need for mini-Quasars as early sources of re-ionising photons? (1)
\item According to Section 4.1.1, how does the formation of the first galaxies depend on the assumed cosmological parameters? (1)
\item Provide an account of the problems associated with forming a {\it supermassive} black hole from a Pop {\sc iii} seed. (3)
\item By editing the Python notebook associated with Lecture 7 to generate the relevant plot, show that the approximation $d_L\sim100[(1+z)/10]$ given in Section 6.2 is reasonable (you should include your plot in your answer) (3).
\item Explain why Lyman-$\alpha$ from galaxies at $z>10$ would be severely attenuated if the bulk of the Universe was neutral at high redshifts? (2) 
\end{enumerate}

\end{document}